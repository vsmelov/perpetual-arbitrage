\documentclass{article}
\usepackage[left=3cm,right=3cm,top=2cm,bottom=2cm]{geometry} % page settings
\usepackage{fourier}
\usepackage{listings}
\usepackage{amsmath} % provides many mathematical environments & tools

\setlength{\parindent}{0mm}

\begin{document}

\title{Perpetual arbitrage}
\author{Vladimir Smelov}
\date{\today}
\maketitle

\section{Abstract}
This article explains how to do direct/inverse perpetual contracts arbitrage keeping portfolio delta-neutral and having exposure in USDT.

\section{PNL and Delta formulas}

For a direct BTCUSDT contract

\begin{align*}
Profit^{BTCUSDT}_{linear} = CurrentPrice^{BTC} - AvgEntryPrice^{BTC}
\end{align*}

So the contract derivatives are

\begin{align*}
& \Delta^{BTCUSDT}_{linear} = \frac{\partial Profit^{USDT}_{linear}}{\partial CurrentPrice^{BTC}} = 1 \\
& \Gamma^{BTCUSDT}_{linear} = \frac{\partial \Delta^{BTCUSDT}_{linear}}{\partial CurrentPrice^{BTC}} = 0
\end{align*}


For an inverse BTCUSD contract

\begin{align*}
& Profit^{BTC}_{inverse} = \frac{1}{AvgEntryPrice^{BTC}} - \frac{1}{CurrentPrice^{BTC}} \\
& Profit^{USD}_{inverse} = CurrentPrice^{BTC} * Profit^{BTC}_{inverse} \\
& Profit^{USD}_{inverse} = CurrentPrice^{BTC} * (\frac{1}{AvgEntryPrice^{BTC}} - \frac{1}{CurrentPrice^{BTC}}) \\
& Profit^{USD}_{inverse} = \frac{CurrentPrice^{BTC}}{AvgEntryPrice^{BTC}} - 1 \\
\end{align*}

So the contract derivatives are

\begin{align*}
& \Delta^{BTCUSD}_{inverse} = \frac{\partial Profit^{USD}_{inverse}}{\partial CurrentPrice^{BTC}} = \frac{1}{AvgEntryPrice^{BTC}} \\
& \Gamma^{BTCUSD}_{inverse} = 0
\end{align*}

Keep in mind that our profit measured in BTC also has delta by the nature equals to 1.

So total portfolio delta equals to:


\begin{align*}
& \Delta Porfolio = Position^{BTCUSDT}_{linear} * \Delta^{BTCUSDT}_{linear} + Position^{BTCUSD}_{inverse} * \Delta^{BTCUSD}_{inverse} + 1 * Account^{BTC} \\
& \Delta Porfolio = Position^{BTCUSDT}_{linear} + Position^{BTCUSD}_{inverse} * \frac{1}{AvgEntryPrice^{BTC}} + Account^{BTC} \\
\end{align*}

\danger We assuming USDT and USD absolutely aligned, what is not completely true. We should handle this in future.





\section{Trade effect on delta position}

Let's say, our position is 
$Position^{BTCUSDT}_{inverse} = s1$ entered for $AvgEntryPrice^{BTC} = p1$ \\

We do a trade with $TradeSize = s2$ and $TradePrice = p2$
\\

What will be the new value of $AvgEntryPrice^{BTC} = x$ \\

Let's say we will close the total position later, when the price will be $p3$ \\

Then PNL of closing (s1+s2)@p3 with $AvgEntryPrice^{BTC} = x$ should be equals to PNL of closing s1@p3 with $AvgEntryPrice^{BTC} = p1$ + closing s2@p3 with $AvgEntryPrice^{BTC} = p2$ \\

\begin{align*}
& s1 * (1/p1 - 1/p3) + s2 * (1/p2 - 1/p3) = (s1+s2) * (1/x - 1/p3) \\
& s1 * 1/p1 + s2 * 1/p2 = (s1+s2) * 1/x \\
& s1 * 1/p1 + s2 * 1/p2 = (s1+s2) * 1/x \\
& x = \frac{p1 p2 (s1 + s2)}{p1 s2 + p2 s1} \\
\end{align*}

Let's note that it's the harmonical average formula. \\

So after a trade our delta position on inverse perpetual will be

\begin{align*}
& \Delta_{NEW} = (s1 + s2) * 1 / x \\
& = (s1 + s2) \frac{p1 s2 + p2 s1}{p1 p2 (s1 + s2)} \\
& = \frac{p1 s2 + p2 s1}{p1 p2} \\
& = \frac{PreviousAverageEntryPrice * TradeSize + TradePrice * PreviousPosition}{PreviousAverageEntryPrice * TradePrice} \\
& = \frac{PreviousPosition}{PreviousAverageEntryPrice} + \frac{TradeSize}{TradePrice} \\ 
& = \Delta PreviosPosition + \Delta Trade
\end{align*}

so delta-position is additive what is logical :-) \\

\subsection{Long-Short effect on delta position}

What will happen if we, say, long s@p1, then short s@p2 using an inverse perpetual? \\

Our current position is 0. \\

Our BTC profit is $ s * (1/p1-1/p2) $ \\

Our USD profit at the moment of the trade is $ p2 * s * (1/p1-1/p2) = s * (p2/p1 - 1) $ \\

Our USD delta is $ \frac{s}{p1} - \frac{s}{p2} $ \\

How is it turned possible that position is 0, but Delta is not 0? \\

It happens because Long-Short trade generated the BTC profit and it has Delta by itself! \\

In a trading-bot it's better to separate btc-wallet-profit from position and keep AvgEntryPrice for position.

\section{Limit wallet BTC amount}

After we make a lot of successful trades with inverse perpetual, we'll get a lot of BTC as a profit on the account.

The problem is that BTC is very volatile by itself so our profit is volatile.

To avoid this and have exposure in USDT we have two options:

\subsection{withdraw BTC}

Just withdraw BTC manually, exchange it to USDT and send it to linear exchange account :-)

\subsection{cross-exchanges transfer BTC to USDT with trades}

The idea is to use situations when we can trade with total profit = 0, making profit trades with linear perpetual and making equal loss with inverse perpetual. \\

Let's say market prices are distributed around \$1000. \\

Let LinearPrice = \$1100, InversePrice = \$1100, if we do SHORT linear, LONG inverse, and then when prices become normal (near \$1000), we close positions, we'll loose 0.1btc on inverse perpetual, but win \$100 on linear, so this way "cross-exchanges transfering" happen! \\

If we wait for situations when LinearPrice gap even higher than InversePrice we'll even make some total profit. \\

In real situation, for a good model, market prices are distributed around theoretical price. So the logic remains the same. \\

\warning If both prices linear and inverse are higher than theoretical, it may indicate up-trend. Nothing really bad may happen, taking into account that our TotalPosition is delta-neutral and for a good model this should be a rare case (good model makes correct predictions with a higher probability than incorrect). \\

\subsection{BTC wallet limit}

When we see that WalletInverseBTC is high enough (and before we resolve it with one of the methods above), we disable trades which tend to increase InverseProfit. \\

IF WalletInverseBTC > Max: \\
DO NOT LONG when TheoPriceInverse > MarketPriceInverse \\
DO NOT SHORT when TheoPriceInverse < MarketPriceInverse \\

\section{Arbitrage algorithm idea}

Let's say, we have a good model which gives predictions of TheoPrice perpetual value. \\

Let's make an algorithm to trade using the simple criteria: \\

CRITERA = (MarketPriceLinear - TheoPriceLinear) - (MarketPriceInverse - TheoPriceInverse)

\danger different exchanges may have different funding mechanisms, different involved risk and different fees, so strictly speaking, this can affect theoprice.

IF CRITERIA > SENSITIVITY: SHORT linear, LONG inverse \\
IF CRITERIA < -SENSITIVITY: LONG linear, SHORT inverse \\

After we do trades, we always hedge WalletBTC with short/long linear perpetual, so PortfoiloDelta is always 0.

If WalletBTC is high we disable further increasing of BTC (see "BTC wallet limit"), and transfer BTC to USDT account with the method described in "cross-exchanges transfer BTC to USDT with trades".

\section{Arbitrage algorithm example}

Let's say market prices are normally distributed around \$1000.

Let SENSITIVITY=1.

\begin{lstlisting}
for i in range(20):
    print(f'time={i}, linear={int(random.gauss(1000, 10))}, inverse={int(random.gauss(1000, 10))}')

time=0, linear=1005, inverse=994
# CRITERA = (1005-1000) - (994-1000) = 11 -> SHORT linear, LONG inverse
# short 1btc@1005, long 994usdt@994

time=1, linear=1004, inverse=1011
# CRITERA = -6 -> LONG linear, SHORT inverse
# long linear 1btc@1004, short inverse -994usdt@1011,  
# linear profit = -1usdt
# inverse profit = 994*(1/994-1/1011) = 0.017btc
# short linear -0.017btc@1004 to hedge

time=2, linear=989, inverse=995
# CRITERA = -6 -> LONG linear, SHORT inverse
# opportunity to "cross-exchange-transfer BTC" with making money on linear
# short inverse -995usdt@995
# long linear 1btc@989, so linear position is +1-0.017=0.983 with 
# linear profit -0.017*(989-1004)=0.255usdt
# so linear position 0.983btc@989, inverse position -995usdt@995, inverse profit = 0.017btc, linear profit=-1+0.255=-0.745usdt


time=3, linear=1002, inverse=1000
# CRITERA = 2 -> SHORT linear, LONG inverse
# short linear -0.983btc@1002, linear profit += 0.983*(1002-989)=12.779usdt
# long inverse 995usdt@1000, inverse profit += -995*(1/995-1/1000) = -0.005btc
# note, that total usd profit 12.779-0.005*1000=7.779 > 0
# we could do trades on time=2 for even bigger size and decrease BTC wallet even more
# linear profit = -0.745+12.779 = 12usdt, inverse profit = 0.017-0.005 = 0.012btc
# to hedge we do short -0.012btc@1002 <- linear position
# and inverse position = 0usdt

time=4, linear=987, inverse=1002
# CRITERIA = -15 -> DO short inverse, long linear

time=5, linear=998, inverse=1005
# CRITERIA = -7, do not do anything, wait to close position

time=5, linear=1000, inverse=1003
# CRITERIA = -3, do not do anything, wait to close position

time=6, linear=993, inverse=999
# CRITERIA = -6, do not do anything, wait to close position

time=7, linear=1010, inverse=1004
# CRITERIA = 6, DO long inverse, short linear
# make money on closing the position
# hedge WalletBTC delta with linear perpetual

time=8, linear=1012, inverse=1011
# if WalletBTC is big enough we can decrease it \\
# using "cross-exchange-transfer BTC" method (see above)

...
\end{lstlisting}

\section{Appendix}
Exchanges with perpetual contracts:

\begin{itemize}
\item DueDEX (BTCUSDT linear)
\item BitMEX (BTCUSD inverse)
\item Deribit (BTCUSD inverse)
\item OceanEX (both)
\item ByBit (both)
\item Poloniex(BTCUSDT linear)
\item CoinBene (various direct PPT)
\item ...
\end{itemize}


Open questions:
\begin{itemize}
\item Why exactly there are no linear USD perpetuals? Is there some specific law or something?
\end{itemize}


\section{References}
\begin{itemize}
\item https://duedex.zendesk.com/hc/en-us/articles/360046896273-DueDEX-USDT-Perpetual-Contract-FAQ
\item https://blockonomi.com/perpetual-contracts-bitcoin/
\item https://forklog.com/likvidatsiya-marzha-i-usdt-kontrakty-torguem-na-bybit/
\item https://www.bitmex.com/app/pnlGuide
\item https://help.bybit.com/hc/en-us/articles/900000306086
\end{itemize}


\end{document}